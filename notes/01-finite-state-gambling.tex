\documentclass[../main.tex]{subfiles}

\begin{document}

Let $\Sigma$ be an alphabet with $2 \leq |\Sigma| < \infty$. That is, $\Sigma$
is finite.
A probability measure on $\Sigma$ is a function:
\begin{align*}
    & \pi: \Sigma \rightarrow [0, \infty)
\end{align*}
satisfying:
\begin{align*}
    & \sum_{a \in \Sigma} \pi(a) = 1
\end{align*}
A rational probability measure on $\Sigma$ is a function:
\begin{align*}
    & \pi: \Sigma \rightarrow \Q \cap [0, \infty)
\end{align*}
that is a probability measure on $\Sigma$.
We write:
\begin{align*}
    & \Delta(\Sigma) = \{ \text{probability measures on $\Sigma$} \} \\
    & \Delta_{\Q} = \{ \text{rational probability measures on $\Sigma$} \} \\
    & \Delta^{+}(\Sigma) = \{ \pi \in \Delta(\Sigma) | (\forall a \in \Sigma) \pi(a) > 0 \} \\
    & \Delta^{+}_{\Q}(\Sigma) = \{ \pi \in \Delta_{\Q}(\Sigma) | (\forall a \in \Sigma) \pi(a) > 0 \} \\
\end{align*}
Note that $\Delta(\Sigma)$ is a $(|\Sigma| - 1)$-dimensional simplex in $\R^{|\Sigma|}$.

\begin{defn}
    A finite-state automaton (FSA) on $\Sigma$ is a triple:
    \begin{align*}
        A = (Q, \delta, s)
    \end{align*}
    where:
    \begin{align*}
        & Q \text{ is a finite set of states,} \\
        & \delta:Q\times\Sigma \rightarrow Q \text{ is a transition function,} \\
        & s \in Q \text{ is a start state.}
    \end{align*}
\end{defn}

\begin{exmp}
    DIAGRAM
\end{exmp}

Given an FSA, $A = (Q, \delta, s)$ on $\Sigma$, we define the extended-transition
function:
\begin{align*}
    & \hat{\delta}:Q\times\Sigma^{*} \rightarrow Q
\end{align*}
by the recursion:
\begin{align*}
    & \hat{\delta}(q, \lambda) = q \\
    & \hat{\delta}(q, wa) = \delta(\hat{\delta}(q, w), a)
\end{align*}
$\forall q \in Q, w \in \Sigma^{*}$, and $a \in \Sigma$.

\textbf{Notational Conventions:}
\begin{enumerate}
    \item We write $\delta(q, w)$ for $\hat{\delta}(q, w)$.
    \item We write $\delta(w)$ for $\delta(s, w)$.
\end{enumerate}

Our next objective is to endow FSAs with the ability to gamble.

\begin{defn}
    A bet on $\Sigma$ is a rational probability measure $\beta \in \Delta_{\Q}(\Sigma)$.
\end{defn}

\textbf{Intuition:}
Assume that:
\begin{itemize}
    \item You have $d \in \Q \cap [0, \infty)$ dollars
    \item You are confronting an experiment whose outcome is some element of $\Sigma$
    \item You place the bet $\beta \in \Delta_{\Q}(\Sigma)$
\end{itemize}

This means that $\forall a \in \Sigma$, you are betting $d\cdot\beta(a)$ dollars
that the outcome is $a$. Since $\sum_{a \in \Sigma} d\beta(a) = d$, you have to
bet all your money in this scenario. After the bet, if the martingale's outcome
was $a$, you will have $d(a)$, an amount that we now specify.

Suppose that the outcomes of this experiment occur according to a probability
measure $\pi \in \Delta(\Sigma)$. What is your expected amount of money after
the bet, given that you had $d$ dollars?

Answer:
\begin{equation*}
    \sum_{a \in \Sigma} \pi(a)d(a) = \mathbb{E}_{\pi}[d(a) | d]
\end{equation*}

Therefore ``the payoffs are fair'' if:
\begin{equation*}
    d = \sum_{a \in \Sigma} \pi(a)d(a)
\end{equation*}

\textbf{Recall:}
A bet on $\Sigma$ is a probability measure $\beta \in \Delta_{\Q}(\Sigma)$.
For now, a payoff rule on $\Sigma$ is a probability measure $\rho \in \Delta(\Sigma)$.

\textbf{Intuition:}
    Assume that a gambler has $d \in [0, \infty)$ dollars and places a bet,
$\beta$ on an experiment that will have an outcome that is an element of $\Sigma$.
Placing this bet means that, $(\forall a \in \Sigma)$, the gambler is betting
$d\beta(a)$ dollars that the outcome is $a$. Note that $d = \sum_{a \in \Sigma} d\beta(a)$,
so the gambler is required to bet all its money.

    If the payoff rule is $\rho \in \Delta(\Sigma)$, and the actual outcome is
$a \in \Sigma$, then the gambler will have:
\begin{equation}
    d(a) := \frac{\beta(a)}{\rho(a)}
\end{equation}
dollars after the bet.

    Now assum that the outcome of the experiment occurs according to a probabilty
measure, $\pi \in \Delta(\Sigma)$, which we call the actual probability measure
of the experiment. The expected value of the gambler's amount of money after the
bet (given that it has $d$ dollars before the bet) is:
\begin{equation}
    \mathbb{E}_{a\sim\pi}[d(a)] := \sum_{a \in \Sigma} d(a)\pi(a),
\end{equation}
where $a\sim\pi$ means ``a is drawn according to $\pi$''. By (2.1) and (2.2),
\begin{equation}
    \mathbb{E}_{a\sim\pi}[d(a)] = d \sum_{a \in \Sigma} \frac{\beta(a)\pi(a)}{\rho(a)}
\end{equation}

\textbf{Observation:}
If $\rho = \pi$, i.e., the payoff rule is the actual probability measure on $\Sigma$,
then the payoffs are fair in the sense that $\mathbb{E}_{a \sim \pi}[d(a)] = d$.

\begin{defn}
    A finite state gambler (FSG) on $\Sigma$ is a 5-tuple, $G = (Q, \delta, s, \beta, c)$, where:
    \begin{itemize}
        \item $(Q, \delta, s)$ is a FSA,
        \item $\beta: Q \rightarrow \Delta_{\Q}(\Sigma)$ is the betting function,
        \item $c \in \Q \cap [0, \infty)$ is the initial capital.
    \end{itemize}
\end{defn}

\begin{exmp}
    DIAGRAM
\end{exmp}

\textbf{Intuition:}

    Before defining the semantics of FSGs formally, let us use example 2.5 to
gain some insight. Assume that $G$ is given an input string $w \in \{0, 1\}^{*}$,
whose bits are chosen by independent tosses of a fair coin, and asusme that the
payoff rule coincides with this. Let $d_{G}(w)$ denote the amount of money that
$G$ has after betting on $w$. Then:
\begin{align*}
    & d_{G}(\lambda) = c = 1 \\
    & d_{G}(1) = \frac{4}{3} = 2\frac{2}{3}d_{G}(\lambda) \\
    & d_{G}(11) = 2\frac{1}{3}d_{G}(1) = \frac{8}{9} \\
    & d_{G}(110) = 2\frac{2}{3}d_{G}(11) = \frac{32}{27} \\
    & ...
\end{align*}

\begin{defn}
        If $G = (Q, \delta, s, \beta, c)$ is an FSG and $\pi \in \Delta^{+}(\Sigma)$,
    then the $\pi$-martingale of $G$ is the function:
    \begin{equation*}
        d^{\pi}_{G}: \Sigma^{*} \rightarrow [0, \infty)
    \end{equation*}
    defined by the recursion:
    \begin{align*}
        & d^{\pi}_{G}(\lambda) = c \\
        & d^{\pi}_{G}(wa) = d^{\pi}_{G}(w) \frac{\beta(\delta(w))(a)}{\pi(a)}
    \end{align*}
\end{defn}

\begin{defn}
    (Ville - 1939). Let $\pi \in \Delta(\Sigma)$. A $\pi$-martingale on $\Sigma$
    is a function:
    \begin{equation*}
        d: \Sigma^{*} \rightarrow [0, \infty)
    \end{equation*}
    that satisfies, $(\forall w \in \Sigma^{*})$:
    \begin{equation}
        d(w) = \sum_{a \in \Sigma} d(wa)\pi(a)
    \end{equation}
\end{defn}

\textbf{Obervation:}
    For every FSG G on $\Sigma$ and every probability measure $\pi \in \Delta^{+}(\Sigma)$,
$d^{\pi}_{G}$ is a $\pi$-martingale.

\textbf{Intuition:}
We work in the sequence space $\Sigma^{\infty}$ of all (infinite) sequences
    $S = a_0 a_1 a_2 ... $
of symbols in $\Sigma$. For each $S \in \Sigma^{\infty}$ and $m, n \in \N$,
we use the notation $S[m..n]$ for the string consisting of the $m$th through
$n$th symbols in $S$.
\begin{enumerate}
    \item $S[0]$ is the left most symbol in $S$.
    \item If $m > n$, then $S[m..n] = \lambda$.
\end{enumerate}
We call a string $w \in \Sigma^{*}$ a prefix of $S \in \Sigma^{\infty}$ if
$S[0..|w|-1] = w$. Recall that, for $\pi \in \Delta(\Sigma)$, a $\pi$-martingale
on $\Sigma$ is a function $d:\Sigma^{*} \rightarrow [0, \infty)$ such that:
\begin{equation*}
    d(w) = \sum_{a \in \Sigma} d(wa) \pi(a)
\end{equation*}

\begin{defn}
    Let $\pi \in \Delta(\Sigma)$ and let $d$ be a $\pi$-martingale on $\Sigma$.
    \begin{enumerate}
        \item $d$ succeeds on a sequence $S \in \Sigma^{\infty}$ if:
            \begin{equation*}
                \limsup\limits_{n \rightarrow \infty} d(S[0..n-1]) = \infty
            \end{equation*}
        \item $d$ succeeds strongly on a sequence $S \in \Sigma^{\infty}$ if:
            \begin{equation*}
                \liminf\limits_{n \rightarrow \infty} d(S[0..n-1]) = \infty
            \end{equation*}
        \item The success set of $d$ is:
            \begin{equation*}
                S^{\infty}[d] = \{ S \in \Sigma^{\infty} | d \text{ succeeds on } S \}
            \end{equation*}
        \item The success set of $d$ is:
            \begin{equation*}
                S^{\infty}_{str}[d] = \{ S \in \Sigma^{\infty} | d \text{ succeeds strongly on } S \}
            \end{equation*}
    \end{enumerate}
\end{defn}

\begin{defn}
    The cylinder generated by a string $w \in \Sigma^{*}$ is the set:
    \begin{equation*}
        w\Sigma^{\infty} = \{ S \in \Sigma^{\infty} | w \sqsubseteq	S \}
    \end{equation*}
\end{defn}

\textbf{Intuition:}
If we choose the symbols in a sequence $S \in \Sigma^{\infty}$ independently
according to $\pi$, what is the probability that some given $w$ is a prefix
of $S$?
\begin{equation*}
    \prod_{i=0}^{|w| - 1} \pi(w[i])
\end{equation*}
Call this $\pi(w)$. Now $\pi: \Sigma^{*} \rightarrow [0, 1]$.

\begin{defn}
    A set $Z \subseteq \Sigma^{\infty}$ has $\pi$-measure 0, and we write
    $\mu[\pi](Z) = 0$ if, $(\forall \epsilon > 0)$, there is a function
    $g: \N \rightarrow \Sigma^{*}$ with the following two properties:
    \begin{enumerate}[(i)]
        \item $Z \subseteq \bigcup\limits_{n=0}^{\infty} g(n)\Sigma^{\infty}$
        \item $\sum_{n=0}^{\infty} \pi(g(n)) < \epsilon$
    \end{enumerate}
\end{defn}

\begin{defn}
    An ideal on $\Sigma^{\infty}$ is a non-empty collection, $\mathscr{I}$, of
    subsets of $\Sigma^{\infty}$ such that the following two conditions hold
    for all sets $X, Y \subseteq \Sigma^{\infty}$:
    \begin{enumerate}
        \item $X \subseteq Y \in \mathscr{I} \Rightarrow X \in \mathscr{I}$
        \item $X, Y \in \mathscr{I} \Rightarrow X \cup Y \in \mathscr{I}$
    \end{enumerate}
\end{defn}

An ideal $\mathscr{I}$ is proper if $\Sigma^{\infty} \neq \mathscr{I}$. An ideal
$\mathscr{I}$ is non-atomic if, $\forall s \in \Sigma^{\infty}$,
$\{s\} \in \mathscr{I}$.

\textbf{Intuition:}
An ideal is a notion of smallness for subsets of $\Sigma^{\infty}$. Condition 1
says that subsets of small sets are small. Condition 2 says that the union
of two small sets is small (whence small sets are very small). An ideal is proper
if not every set is small, and it is non-atomic if every singleton set is small.

\begin{exmp}
    \begin{enumerate}
        \item $\{ \emptyset \}$ is the smallest ideal. That is, it is an ideal
            and it is a subset of every ideal. Claim: $\emptyset$ is in every
            ideal by $(1)$.
        \item $\mathscr{P}(\Sigma^{\infty})$ is the largest ideal, which has to
            contain every ideal, and it is not proper.
        \item For every set, $X \subseteq \Sigma^{\infty}$, $\mathscr{P}(X)$ is
            an ideal on $\Sigma^{\infty}$.
        \item $FIN = \{ z \subseteq \Sigma^{\infty} | z \text{ is finite } \}$
            is a proper, nonatomic ideal on $\Sigma^{\infty}$.
        \item $CTBL = \{ z \subseteq \Sigma^{\infty} | z \text{ is countable } \}$
            is a proper, nonatomic ideal on $\Sigma^{\infty}$.
    \end{enumerate}
\end{exmp}

\begin{defn}
    A $\sigma$-ideal on $\Sigma^{\infty}$ is an ideal $\mathscr{I}$ on
    $\Sigma^{\infty}$ that is closed under countbale unions, i.e.,
    \begin{equation*}
        Z_0, Z_1, Z_2, ... \in \mathscr{I} \Rightarrow \bigcup\limits_{n=0}^{\infty} Z_n \in \mathscr{I}
    \end{equation*}
\end{defn}

\textbf{Recall:}
Let $\pi \in \Delta(\Sigma)$. A set $Z \subseteq \Sigma^{\infty}$ has $\pi$-measure
0 and we write $\mu[\pi](Z) = 0$, if, for every $\epsilon > 0$, there is a function
$g:\N \rightarrow \Sigma^{*}$ with the following two properties:
\begin{enumerate}
    \item $Z \subseteq \bigcup\limits_{n = 0}^{\infty} g(n) \Sigma^{\infty}$
    \item $\sum_{n=0}^{\infty} \pi(g(n)) < \epsilon$
\end{enumerate}

\begin{thm}
    Let $\pi \in \Delta(\Sigma)$. The collection of $\pi$-measure 0 subsets of
    $\Sigma^{\infty}$ is a proper $\sigma$-ideal on $\Sigma^{\infty}$. It is
    nonatomic if and only if $\pi \in \Delta^{+}(\Sigma^{\infty})$.
\end{thm}
\begin{proof}
    (scratch)
\end{proof}

\begin{problem}{2.1}
    Let $\Sigma = \{ 0, 1, ... , b - 1 \}$ where $b \geq 2$, and let $\pi$ be
    the uniform probability measure on $\Sigma$, i.e., $\pi(a) = \frac{1}{b}$,
    $\forall a \in \Sigma$. For each of the following two sets, $X$, desgin a
    $FSG$, $G$, such that $X \subseteq S^{\infty}[d^{\pi}_{G}]$.
    \begin{enumerate}[(a)]
        \item $X = \{ S \in \Sigma^{\infty} | (\forall^{\infty} n \in \N) S[n] > 0 \}$
        \item $X = \{ S \in \Sigma^{\infty} | (\forall^{\infty} n \in \N) S[3n] = S[3n + 1] \}$
    \end{enumerate}
\end{problem}

\begin{problem}{2.2}
    Let $\Sigma$ and $\pi$ be as in problem $2.1$. Prove that, for all
    $\pi$-martingales, $d$, there is a $\pi$-martingale, $\hat{d}$ such that:
    \begin{equation*}
        S^{\infty}[d] \subseteq S^{\infty}_{str}[\hat{d}]
    \end{equation*}
\end{problem}


\end{document}
